\documentclass[11pt]{article}   % Tipo de documento
\usepackage[utf8]{inputenc}     % Uso de la codificación UTF-8
\usepackage{comment}            %Se importa el uso de comentarios multilínea

\begin{comment}
    Esto sería un comentario multilínea
\end{comment}

\title{Primer documento de prueba}   % Título se que importa a \maketitle
\author{Cristóbal José Jiménez Gómez % Autor; todo esto se importa a \maketitle
    \thanks{A mi familia, amigos y compañeros de la universidad.}}  %agradecimientos 
\date{25/02/2023} %Se peude poner la fecha de forma manual
\date{\today} % Cada vez que se hace un build se genera la fecha; estas 2 se importan a \maketitle

\begin{document}    %Empieza el documento

    \begin{titlepage}   %Empieza el título
        \maketitle
    \end{titlepage}     %Sacaba el título

    %Párrafo normal
    Esto sería aun primer párrafo rollo de chilleito, ya tu sabe.

    %Párrafo de resumen
    \begin{abstract}
        Esto rpetende ser un texto un poco más formal. El cual contiene un resumen de cosa.
    \end{abstract}

    %Parrafo normal
    Y aquí voy a partir una línea, porque puedo y porque \\ quiero

    %Distintos tipos de formato
    Formatos:\\
    - Para hacer letras en \textbf{negrita} se usará textbf(palabra).\\
    - Para hacer que sean \underline{subrayadas} se usará underline(palabra).\\
    - Para hacer que sean \textit{cursiva} se usará textit(palabra).\\
    - Con \emph{énfasis} será negrita y cursiva: emph(palara) - (para resaltar el texto en el formato en elq ue se encuentra)


    Ver los distintos tipos de listas aquí \url{video que lo explica}{https://www.youtube.com/watch?v=fZiOXWpRNTE}
    %Listas
    Listas:\\
    Lista enumerada:
    \begin{enumerate}
        \item Item 1
        \item Item 2
        \begin{enumerate}
            \item Item 2.1
            \item Item 2.2
            \begin{enumerate}
                \item Item 2.2.1
                \item Item 2.2.2
                \begin{enumerate}
                    \item Item 2.2.2.1
                    \item Item 2.2.2.2
                \end{enumerate}
            \end{enumerate}
            \item El mejor
        \end{enumerate}
    \end{enumerate}


    Lista no enumerada:
    \begin{itemize}
        \item Item 1
        \item Item 2
        \begin{itemize}
            \item Item 2.1
            \item Item 2.2
            \begin{itemize}
                \item Item 2.2.1
                \item Item 2.2.2
                \begin{itemize}
                    \item Item 2.2.2.1
                    \item Item 2.2.2.2
                \end{itemize}
            \end{itemize}
            \item El mejor
        \end{itemize}
    \end{itemize}
\end{document}  %Saacaba el documento